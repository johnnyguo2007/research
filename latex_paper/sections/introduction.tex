% !TeX root = ../article.tex
\section*{Introduction}
The urban heat island (UHI) effect, a phenomenon characterized by elevated temperatures in urban areas compared to their surrounding rural areas, is a concern exacerbated by the escalating effects of global warming and rapid urbanization. When compounded by heat waves (HWs), this effect is particularly worrisome, creating a synergistic impact that can strain urban infrastructure, public health, and energy resources\unskip~\cite{2755510:33598912,2755510:33598911} . A better understanding of the interaction between UHI and HW will lead to the development of effective mitigation and adaptation strategies.

A considerable body of research exists on UHI-HW interactions\unskip~\cite{2755510:33598930}. Heatwaves can modulate the spatial and temporal dynamics of the UHI through surface energy balance modifications, altered atmospheric processes, and changes in anthropogenic heat emissions. While several studies document synergistic relationships between UHI and HW events, others have not found such reinforcing effects. Most studies are limited to single cities or specific geographical regions\unskip~\cite{2755510:33598950,2755510:33598949,2755510:33598945,2755510:33598943,2755510:33598941,2755510:33598938,2755510:33598937,2755510:33598935}. With varying turbulent flux patterns, diverse climatological variables, land surface characteristics, and anthropogenic drivers contributing to the UHI-HW interaction, a global study will provide a more comprehensive understanding of the impacts of those underlying factors. Moreover, many studies rely on satellite-derived land surface temperatures (LST) to examine UHIs, which, while offering wide coverage and temporal consistency, do not directly reflect the thermal conditions experienced by people within urban areas\unskip~\cite{2755510:33598945,2755510:33598947} . The Canopy Urban Heat Island, measured using air temperature, is more consistent with the thermal environment experienced by urban populations, making it a more pertinent indicator of heat stress and its direct impact on public health and thermal comfort\unskip~\cite{2755510:33598934} .

This study addresses these Johnny research gaps by conducting a global-scale analysis of UHI-HW interactions using 2-m air temperature. It will characterize global spatial and temporal patterns of UHI-HW interactions, identifying their variability across climate zones and throughout the day. It will also analyze the relative importance of drivers such as humidity, wind, and energy budget factors across different regions. 