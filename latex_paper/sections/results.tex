% !TeX root = ../article.tex
\section*{Results}

\bgroup
\fixFloatSize{images/250138a4-7f30-44cb-bdb6-c2c5f8875d39-ufigure_1_global_mean_uhi_by_hour.png}
\begin{figure*}[!htbp]
\centering \makeatletter\IfFileExists{images/250138a4-7f30-44cb-bdb6-c2c5f8875d39-ufigure_1_global_mean_uhi_by_hour.png}{\includegraphics{images/250138a4-7f30-44cb-bdb6-c2c5f8875d39-ufigure_1_global_mean_uhi_by_hour.png}}{\includegraphics{250138a4-7f30-44cb-bdb6-c2c5f8875d39-ufigure_1_global_mean_uhi_by_hour.png}}
\makeatother 
\caption{{\textbf{Diurnal composite of UHI and HW interaction for all global cities.} The solid line represents the mean, and the shaded area represents +- one standard deviation.}}
\label{fig:global_diurnal_uhi_hw_interaction}
\end{figure*}
\egroup




\subsection*{Global and regional patterns of UHI-HW interactions }The global simulation results show a clear hourly temporal pattern for the synergy of UHI and HWs. The NW-NHW UHI exhibits a peak at around 05 to 06 and a trough at 09 to 10 local solar time. It also shows a local peak at around 19 local solar time (Figure~\ref{fig:global_diurnal_uhi_hw_interaction}). Aggregating to the diurnal level, we found the nighttime average  \ensuremath{\Delta }UHI 0.27 \ensuremath{\pm} 0.23 \ensuremath{^\circ}C (mean \ensuremath{\pm} 1 s.d.) is higher than its daytime counterpart 0.02\ensuremath{\pm} 0.22\ensuremath{^\circ}C, even though neither result is statistically significant from zero (P {\textgreater} 0.05 for both cases). The results are consistent with previous findings, where positive \unskip~\cite{2755510:33598935,2755510:33598909,2755510:33598908,2755510:33598952} , insignificant\unskip~\cite{2755510:33598915} , and negative \unskip~\cite{2755510:33598943,2755510:33598907,2755510:33598905,2755510:33598896}  \ensuremath{\Delta }UHI were reported, depending on the regions studied.

Figure~\ref{fig:geographical_day_night_duhi} reveals spatial patterns in \ensuremath{\Delta }UHI globally. During the day, notable spatial heterogeneity is observed. Negative or near-neutral interactions are widespread across global coastal areas, major humid mid-latitude agricultural zones, such as the US Corn Belt and Western and Central Europe, and low-latitude tropical regions, including the Congo Basin, Southeast Asia, and southern Mexico. In contrast, distinct positive synergies are clustered within the Indo-Gangetic and North China Plains. At night, the interaction is much stronger and the spatial pattern is more uniform, with the most pronounced hotspot (0.8 \ensuremath{^\circ}C) over northern India. Neutral or slightly negative signals are limited to isolated equatorial grids.


\bgroup
\fixFloatSize{images/9866ade5-4042-4d5f-9574-e00715ca7d29-ufigure_2_ab_all_uhi_day_night_comparison_figure3_style.png}
\begin{figure*}[!htbp]
\centering \makeatletter\IfFileExists{images/9866ade5-4042-4d5f-9574-e00715ca7d29-ufigure_2_ab_all_uhi_day_night_comparison_figure3_style.png}{\includegraphics{images/9866ade5-4042-4d5f-9574-e00715ca7d29-ufigure_2_ab_all_uhi_day_night_comparison_figure3_style.png}}{\includegraphics{9866ade5-4042-4d5f-9574-e00715ca7d29-ufigure_2_ab_all_uhi_day_night_comparison_figure3_style.png}}
\makeatother 
\caption{{\textbf{Nighttime \ensuremath{\Delta }UHI is stronger than that of Daytime.} Geographical distributions of \ensuremath{\Delta }UHI during daytime (a) and nighttime (b). }}
\label{fig:geographical_day_night_duhi}
\end{figure*}
\egroup

\bgroup
\fixFloatSize{images/306ab391-c84b-42e2-af81-6ce3f4ac52e0-ukoppengeiger.png}
\begin{figure*}[!htbp]
\centering \makeatletter\IfFileExists{images/306ab391-c84b-42e2-af81-6ce3f4ac52e0-ukoppengeiger.png}{\includegraphics{images/306ab391-c84b-42e2-af81-6ce3f4ac52e0-ukoppengeiger.png}}{\includegraphics{306ab391-c84b-42e2-af81-6ce3f4ac52e0-ukoppengeiger.png}}
\makeatother 
\caption{{K{\"{o}}ppen{\textendash}Geiger climate map 1991{\textendash}2020\unskip~\protect\cite{2755510:33598889}}}
\label{fig:koppen_geiger_map}
\end{figure*}
\egroup
The geographically-bound patterns observed in Figure~\ref{fig:geographical_day_night_duhi}  show notable alignment with the K{\"{o}}ppen-Geiger climate classification map\unskip~\cite{2755510:33598889}  (Supplementary Figure 1). For example, a distinct negative interaction over the Congo Basin matches the boundaries of the Af (Tropical Rainforest) climate zone. Similar correspondence is evident in regions such as Western Europe, Northeast China, Northern India, Southeast Asia, and the Southeastern United States. The observed agreement between UHI-HW interaction patterns and established climate zones indicates the importance and utility of conducting subsequent analyses stratified by climate classifications.

In this study, we excluded locations belonging to the Polar K{\"{o}}ppen-Geiger climate zone\unskip~\cite{2755510:33598889} . Across the other four major K{\"{o}}ppen-Geiger climate zones, the \ensuremath{\Delta }UHI exhibits a similar hourly temporal pattern to the global one, with early morning peaks and late morning troughs (Figure~\ref{fig:diurnal_duhi_by_climate_zone}). The continental climate zone has the most significant positive peak synergy, while tropical climates have the lowest. Notably, the tropical climate zone is characterized by a significantly higher variance in \ensuremath{\Delta }UHI for both daytime and nighttime, as shown in Figure~\ref{fig:duhi_variance_by_climate_zone}.


\bgroup
\fixFloatSize{images/9124dfda-0a28-451c-b032-6b6a8a81a840-ufigure_4_hourly_plot_4kg.png}
\begin{figure*}[!htbp]
\centering \makeatletter\IfFileExists{images/9124dfda-0a28-451c-b032-6b6a8a81a840-ufigure_4_hourly_plot_4kg.png}{\includegraphics{images/9124dfda-0a28-451c-b032-6b6a8a81a840-ufigure_4_hourly_plot_4kg.png}}{\includegraphics{9124dfda-0a28-451c-b032-6b6a8a81a840-ufigure_4_hourly_plot_4kg.png}}
\makeatother 
\caption{{\textbf{Diurnal Cycle of Heat Wave - Non-Heat Wave Urban Heat Island (\ensuremath{\Delta }UHI) across K{\"{o}}ppen-Geiger Climate Zones.} Line plots showing the mean hourly \ensuremath{\Delta }UHI (\ensuremath{^\circ}C) for Arid, Continental, Temperate, and Tropical climate zones. Shaded areas represent \ensuremath{\pm}1 standard deviation around the mean.}}
\label{fig:diurnal_duhi_by_climate_zone}
\end{figure*}
\egroup

\bgroup
\fixFloatSize{images/9a56a9ee-cad0-4da6-848b-7acd23fc0483-ufigure_5_box_kde_4kg_side_by_side.png}
\begin{figure*}[!htbp]
\centering \makeatletter\IfFileExists{images/9a56a9ee-cad0-4da6-848b-7acd23fc0483-ufigure_5_box_kde_4kg_side_by_side.png}{\includegraphics{images/9a56a9ee-cad0-4da6-848b-7acd23fc0483-ufigure_5_box_kde_4kg_side_by_side.png}}{\includegraphics{9a56a9ee-cad0-4da6-848b-7acd23fc0483-ufigure_5_box_kde_4kg_side_by_side.png}}
\makeatother 
\caption{{\textbf{Continental and temperate climate zones have higher UHI and heatwave interaction. Tropical regions exhibit higher variance. }Each dot represents the average \ensuremath{\Delta }UHI value for each location. The color indicates data density, with yellow indicating high density and navy blue indicating low density. Smooth curves are probability kernel density functions. Box and whiskers show means of \ensuremath{\pm}1 and \ensuremath{\pm} 3 standard deviations.}}
\label{fig:duhi_variance_by_climate_zone}
\end{figure*}
\egroup
To explore why tropical climates exhibit more significant variability in \ensuremath{\Delta }UHI than other climates, this study examined the relationship between \ensuremath{\Delta }UHI and moisture-related factors\ensuremath{^{}}. Because heatwave events are defined as lasting for three or more consecutive days\ensuremath{^{}}, it is possible to analyze the evolution of the UHI-HW interaction for each day throughout an event. This reveals the relationship between environmental conditions and \ensuremath{\Delta }UHI as heatwaves progress.

Figure~\ref{fig:daily_evolution_duhi_by_climate} presents normalized values describing concurrent changes. The normalization was performed by scaling all observed specific humidity and soil moisture at 10 cm depth values linearly to a range between 0 and 1 across all climatic zones. Tropical regions initially demonstrate high humidity and substantial soil moisture (normalized \ensuremath{q_{a}} = 0.8,  \ensuremath{\theta_{10cm}} \ensuremath{\approx} 1.0). These moisture-rich conditions encourage evapotranspiration and cooling in rural areas, leading to an initial \ensuremath{\Delta}UHI of around 0.4\ensuremath{^\circ}C. However, as heatwaves persist, moisture depletion reduces latent heat flux, shifting the energy balance towards sensible heat, thereby diminishing rural cooling effects.


\bgroup
\fixFloatSize{images/978303f5-3f40-483c-a048-931563c73a81-ufigure_6_anova_by_kg_duration.png}
\begin{figure*}[!htbp]
\centering \makeatletter\IfFileExists{images/978303f5-3f40-483c-a048-931563c73a81-ufigure_6_anova_by_kg_duration.png}{\includegraphics{images/978303f5-3f40-483c-a048-931563c73a81-ufigure_6_anova_by_kg_duration.png}}{\includegraphics{978303f5-3f40-483c-a048-931563c73a81-ufigure_6_anova_by_kg_duration.png}}
\makeatother 
\caption{{\textbf{The daily evolution of the difference in urban heat island intensity between heatwave and non-heatwave periods} (\ensuremath{\Delta }UHI, blue line) for four K{\"{o}}ppen-Geiger climate classes{\textemdash}Arid, continental, Temperate, and Tropical. The x-axis indicates the day of the heatwave event (0 = onset), and the left y-axis gives the temperature difference in \ensuremath{^\circ}C. The orange and green lines show the globally normalized near-surface humidity (q\ensuremath{_{a}}) and 10 cm soil moisture (\ensuremath{\Theta}\ensuremath{_{10\ cm}}\noindent ), respectively, plotted against the right y-axis (linearly scale all observed q\ensuremath{_{a}} and \ensuremath{\Theta}\ensuremath{_{10\ cm}} values to the range from 0 to 1). These results illustrate how the additional UHI under heatwaves (blue) relates to concurrent changes in humidity (orange) and soil moisture (green) throughout the event.}}
\label{fig:daily_evolution_duhi_by_climate}
\end{figure*}
\egroup




\subsection*{Global attribution of UHI-HW interactions}
\bgroup
\fixFloatSize{images/074c0709-7f7d-4ca0-a0b5-e99e401de998-ugroup_importance_day_night.png}
\begin{figure*}[!htbp]
\centering \makeatletter\IfFileExists{images/074c0709-7f7d-4ca0-a0b5-e99e401de998-ugroup_importance_day_night.png}{\includegraphics{images/074c0709-7f7d-4ca0-a0b5-e99e401de998-ugroup_importance_day_night.png}}{\includegraphics{074c0709-7f7d-4ca0-a0b5-e99e401de998-ugroup_importance_day_night.png}}
\makeatother 
\caption{{\textbf{SHAP-based analysis of feature group contributions to the predicted \ensuremath{\Delta }UHI}, derived from the CatBoost model trained in this study. The waterfall plots depict each feature group's cumulative percentage contribution to the model's prediction for daytime (a) and nighttime (b) conditions.}}
\label{fig:shap_waterfall_day_night}
\end{figure*}
\egroup

\bgroup
\fixFloatSize{images/0923132c-4c59-4c36-a92b-a97d1d808722-ugroup_summary_day_night.png}
\begin{figure*}[!htbp]
\centering \makeatletter\IfFileExists{images/0923132c-4c59-4c36-a92b-a97d1d808722-ugroup_summary_day_night.png}{\includegraphics{images/0923132c-4c59-4c36-a92b-a97d1d808722-ugroup_summary_day_night.png}}{\includegraphics{0923132c-4c59-4c36-a92b-a97d1d808722-ugroup_summary_day_night.png}}
\makeatother 
\caption{{\textbf{SHAP summary plots showing the distribution and magnitude of feature group impacts}. (a (daytime), b (nighttime)) Each point represents a single instance; the horizontal position indicates the SHAP value (positive values denote an increase in predicted \ensuremath{\Delta }UHI, negative a decrease); the color represents the original feature values (blue: low, red: high).}}
\label{fig:shap_summary_day_night}
\end{figure*}
\egroup
The SHAP waterfall plots (Figure~\ref{fig:shap_waterfall_day_night}) and summary plots (Figure~\ref{fig:shap_summary_day_night}) provide complementary insights into feature group contributions. A feature group consists of all variables of the same metric. The only feature group with more than one feature is q\ensuremath{_{a}}, including both \ensuremath{\Delta }\ensuremath{\delta }q\ensuremath{_{a}} and \ensuremath{\Delta }q\ensuremath{_{a}}. The individual feature's feature importance (Supplementary Figure 1) and summary plots(Supplementary Figure 2) are included in the supplementary material. The summary plots reveal the distribution and directionality of individual feature impacts (positive or negative influence on \ensuremath{\Delta }UHI). \mbox{}\protect\newline In contrast, the waterfall plots quantify the cumulative contribution of each feature to a specific model prediction. A detailed directional contribution analysis will be provided in the discussion section. Net longwave radiation is the most influential factor across daytime(28.20\%) and nighttime(36.47\%) conditions. Sensible heat flux also has consistently high contributions (day: 12.467\%; night: 21.00\%). Specific humidity significantly contributes (19.50\%) during the daytime, while the 10-meter wind is more prominent (16.16\%) at nighttime.



\subsection*{Global attribution by the hour}
\bgroup
\fixFloatSize{images/237443e0-d4d9-432d-a044-28b1fa9a18c4-uhourly_stacked_bar_global.png}
\begin{figure*}[!htbp]
\centering \makeatletter\IfFileExists{images/237443e0-d4d9-432d-a044-28b1fa9a18c4-uhourly_stacked_bar_global.png}{\includegraphics{images/237443e0-d4d9-432d-a044-28b1fa9a18c4-uhourly_stacked_bar_global.png}}{\includegraphics{237443e0-d4d9-432d-a044-28b1fa9a18c4-uhourly_stacked_bar_global.png}}
\makeatother 
\caption{{Global hourly cycle of mean SHAP value contributions. The y-axis represents the mean SHAP value contribution (\ensuremath{^\circ}C). Features' SHAP displayed as stacked bars. The solid black line indicates the total model prediction (sum of all mean SHAP contributions plus the base value), while the red dashed line shows the model's base value (0.186 \ensuremath{^\circ}C). Contributions from individual features (G, L\ensuremath{_{n}}, SH, K\ensuremath{_{n}}, q\ensuremath{_{a}}, AHac, \ensuremath{\Theta}\ensuremath{_{10cm}}, U10), as detailed in the legend, are plotted against local hour.}}
\label{fig:global_hourly_shap_contributions}
\end{figure*}
\egroup
A more granular temporal analysis, as presented in Figure~\ref{fig:global_hourly_shap_contributions}, is important for understanding their dynamics throughout the full diurnal cycle. It illustrates the global hourly cycle of mean SHAP value contributions for these drivers, revealing that their influence is not static. For instance, net longwave radiation (Ln\noindent ) exhibits a positive contribution to the UHI-HW interaction, peaking around 6-7 AM local time and again with a smaller peak around 7-8 PM local time. In contrast, its contribution becomes negative during midday. The influence of specific humidity (q\ensuremath{_{a}}) peaks in the late morning and midday. These detailed hourly patterns, which would be obscured if relying solely on the broader daytime and nighttime averages where the mean \ensuremath{\Delta }UHI was found to be 0.27\ensuremath{\pm}0.23\ensuremath{^\circ}C at night versus 0.02\ensuremath{\pm}0.22\ensuremath{^\circ}C during the day, are essential. Such granularity allows for linking driver influence to the observed overall \ensuremath{\Delta }UHI peaks, such as the one around 5 to 6 AM local time, and troughs, like the one at 9 to 10 AM local time Figure~\ref{fig:global_diurnal_uhi_hw_interaction}. Understanding these specific hourly contributions and transition periods is, therefore, essential for understanding how each variable contributes to the direction of \ensuremath{\Delta }UHI changes.

Figure~\ref{fig:global_hourly_key_feature_shap} and the SHAP summaries in Figure~\ref{fig:shap_summary_day_night}  clarify two important aspects for each driver: whether it enhances or diminishes the \ensuremath{\Delta }UHI and whether changes in its feature values positively or negatively correlate with the SHAP contribution. For net infrared radiation (\ensuremath{\Delta }\ensuremath{\delta }L\ensuremath{_{n}}), larger values consistently correspond to increased positive SHAP contributions in Figure~\ref{fig:global_hourly_key_feature_shap}a, and this positive correlation during both daytime and nighttime is confirmed by the red-shaded bars located on the right-hand side of the X-axis (feature values) in Figure~\ref{fig:shap_summary_day_night}. Similarly, sensible heat flux (\ensuremath{\Delta }SH) provides a positive contribution during the day and a negative contribution during the night(Figure~\ref{fig:global_hourly_key_feature_shap}b). Its SHAP value contribution and feature values generally move in the same direction. Specific humidity (\ensuremath{\Delta }\ensuremath{\delta }q\ensuremath{_{a}}) primarily contributes positively to the UHI from late morning to afternoon(Figure~\ref{fig:global_hourly_key_feature_shap}c). Notably, the SHAP values increase as \ensuremath{\Delta }\ensuremath{\delta }q\ensuremath{_{a}} decreases, indicating a clear negative correlation, which is also confirmed in Figure~\ref{fig:shap_summary_day_night}. The 10 m wind speed (\ensuremath{\Delta }U10) exerts a negative contribution during the night, with SHAP contributions going opposite the change in wind speed values. \ensuremath{\Delta }U10 generally contributes positively during the daytime, though the correlation between feature values and SHAP contribution is less distinct, as observed in Figure~\ref{fig:shap_summary_day_night}  as well. 


\bgroup
\fixFloatSize{images/0be8d0fa-326f-4c43-b7ce-8546c01e3ee5-uglobal_composite_2x3_fira_fsh_q2m_u10_soilwater_10cm_fgr.png}
\begin{figure*}[!htbp]
\centering \makeatletter\IfFileExists{images/0be8d0fa-326f-4c43-b7ce-8546c01e3ee5-uglobal_composite_2x3_fira_fsh_q2m_u10_soilwater_10cm_fgr.png}{\includegraphics{images/0be8d0fa-326f-4c43-b7ce-8546c01e3ee5-uglobal_composite_2x3_fira_fsh_q2m_u10_soilwater_10cm_fgr.png}}{\includegraphics{0be8d0fa-326f-4c43-b7ce-8546c01e3ee5-uglobal_composite_2x3_fira_fsh_q2m_u10_soilwater_10cm_fgr.png}}
\makeatother 
\caption{{\textbf{Global hourly key feature SHAP contributions to \ensuremath{\Delta }UHI and feature values. }Hourly SHAP contributions (bars, \ensuremath{^\circ}C, left y-axis) and mean feature values (lines, right y-axis) for the global climate zone. Panels show: (a) net infrared radiation (\ensuremath{\Delta }\ensuremath{\delta }L\ensuremath{_{n}} in W/m\ensuremath{^2}), (b) sensible heat (\ensuremath{\Delta }SH in W/m\ensuremath{^2}), (c) 2-m specific humidity (\ensuremath{\Delta }\ensuremath{\delta }q\ensuremath{_{a}} in kg/kg ), (d) 10m wind speed (\ensuremath{\Delta }U10 in m/s), (e) 10 cm soil liquid water (\ensuremath{\Delta }\ensuremath{\Theta}10cm), and (f) ground heat flux (\ensuremath{\Delta }\ensuremath{\delta }G in W/m\ensuremath{^2}). Feature colors match Figure~\ref{fig:global_hourly_shap_contributions}.}}
\label{fig:global_hourly_key_feature_shap}
\end{figure*}
\egroup




\subsection*{Regional attribution by the hour}To investigate the regional influences of feature groups, we reutilized the trained CatBoost model. We partitioned the resulting SHAP value contributions into four K{\"{o}}ppen-Geiger climate zones: Arid, Continental, Temperate, and Tropical. Figure~\ref{fig:feature_contribution_by_climate_zone}  illustrates the percentage contribution of each feature group to the predicted \ensuremath{\Delta }UHI within each climate zone.  It reveals several patterns. Net longwave radiation(L\ensuremath{_{n}}) consistently emerges as the most influential factor across all K{\"{o}}ppen{\textendash}Geiger climate zones. Its importance nearly doubles from daytime (around 25{\textendash}30\%) to nighttime (35{\textendash}44\%), highlighting the dominant role of radiative cooling during the nighttime.

Across all studied climate zones during the day ( Figure~\ref{fig:feature_contribution_by_climate_zone} Daytime), aside from L\ensuremath{_{n}}, Specific humidity (q\ensuremath{_{a}}) and 10-meter wind (U10) emerge as dominant contributors to the model's predictions, which is consistent with Figure~\ref{fig:shap_waterfall_day_night}a. The tropical climate is characterized by significant contributions from 2-m specific humidity (q\ensuremath{_{a}}). In this zone, qa contributes much more than any other climate zone. Again, compared to contributions in other climate zones, 10-meter wind (U10, 21\%) has the highest contribution in the arid zone.

At night, the order of importance changes: sensible heat flux (SH) becomes much more impactful and pronounced in continental climates, where SH approaches 23\%. Humidity has smaller contribution after sunset across all climate zones.


\bgroup
\fixFloatSize{images/0dc5196d-0412-4027-8605-13b8c87bef3b-ufigure_7_feature_group_contribution_by_kg_combined.png}
\begin{figure*}[!htbp]
\centering \makeatletter\IfFileExists{images/0dc5196d-0412-4027-8605-13b8c87bef3b-ufigure_7_feature_group_contribution_by_kg_combined.png}{\includegraphics{images/0dc5196d-0412-4027-8605-13b8c87bef3b-ufigure_7_feature_group_contribution_by_kg_combined.png}}{\includegraphics{0dc5196d-0412-4027-8605-13b8c87bef3b-ufigure_7_feature_group_contribution_by_kg_combined.png}}
\makeatother 
\caption{{\textbf{Percentage Contribution of Feature Groups to Predicted \ensuremath{\Delta }UHI, Stratified by K{\"{o}}ppen-Geiger Climate Zones.} Bar chart illustrating the percentage contribution of each feature group to the predicted \ensuremath{\Delta }UHI difference, categorized by K{\"{o}}ppen-Geiger climate zones (Arid, Continental, Temperate, and Tropical). Percentage contributions are derived from aggregated SHAP values, indicating the relative importance of each feature group in driving \ensuremath{\Delta }UHI within each climate region.}}
\label{fig:feature_contribution_by_climate_zone}
\end{figure*}
\egroup
Figure~\ref{fig:q2m_shap_by_climate_zone}   provides a detailed examination of the role of 2-m specific humidity (q\ensuremath{_{a}}) and illustrates its varying impact on \ensuremath{\Delta }UHI across different climate zones. This particular variable is selected because its interaction with \ensuremath{\Delta }UHI has been extensively documented\unskip~\cite{2755510:33598930,2755510:33598952,2755510:33598945,2755510:33598947} . This analysis can help validate previous findings by applying the current framework. 

In Arid zones (Figure~\ref{fig:q2m_shap_by_climate_zone} a), \ensuremath{\Delta }\ensuremath{\delta }qa is generally small and hovers near zero or is slightly positive during the late afternoon and night. It started to turn negative from morning to noon. Correspondingly, the SHAP contribution of q\ensuremath{_{a\ }}in Arid zones is predominantly negative from the morning to noon. However, its SHAP contribution impact is the smallest among the four climate zones studied. It is the only climate zone that shows a material negative contribution during the night. Continental (Figure~\ref{fig:q2m_shap_by_climate_zone} b), Temperate (Figure~\ref{fig:q2m_shap_by_climate_zone} c), and Tropical (Figure~\ref{fig:q2m_shap_by_climate_zone} d) climate zones consistently show negative \ensuremath{\Delta }\ensuremath{\delta }q\ensuremath{_{a}} values. These values indicate drier atmospheric conditions during heatwaves compared to non-heatwave periods in urban areas. A key observation in these four zones is a sharp drop in \ensuremath{\Delta }\ensuremath{\delta }q\ensuremath{_{a}} around the sunrise period (approximately 5-7 AM local time). Concurrently, there is a sharp increase in the positive SHAP contribution of q\ensuremath{_{a}} during this same sunrise period across these zones, with the Tropical zone (Figure~\ref{fig:q2m_shap_by_climate_zone} d) exhibiting the most significant positive SHAP contribution from q\ensuremath{_{a}}.


\bgroup
\fixFloatSize{images/7dcfb871-fb6f-4227-a2c3-01e18b678f80-uq2m_climate_zones_composite_2x2.png}
\begin{figure*}[!htbp]
\centering \makeatletter\IfFileExists{images/7dcfb871-fb6f-4227-a2c3-01e18b678f80-uq2m_climate_zones_composite_2x2.png}{\includegraphics{images/7dcfb871-fb6f-4227-a2c3-01e18b678f80-uq2m_climate_zones_composite_2x2.png}}{\includegraphics{7dcfb871-fb6f-4227-a2c3-01e18b678f80-uq2m_climate_zones_composite_2x2.png}}
\makeatother 
\caption{{\textbf{Diurnal cycle of 2-m specific humidity SHAP contributions and feature differences across distinct climate zones.} The hourly SHAP contributions of 2-m specific humidity (\ensuremath{\Delta }\ensuremath{\delta }q\ensuremath{_{a)\ }}to the \ensuremath{\Delta }UHI (bars, \ensuremath{^\circ}C, left y-axis) alongside the \ensuremath{\Delta }\ensuremath{\delta }q\ensuremath{_{a}} (line with markers, kg/kg , right y-axis). These relationships are presented for four different climate zones: (a) Arid, (b) Continental, (c) Temperate, and (d) Tropical.}}
\label{fig:q2m_shap_by_climate_zone}
\end{figure*}
\egroup 