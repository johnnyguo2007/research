% !TeX root = ../article.tex
\section*{Conclusions}
This global analysis reveals diurnal and spatial patterns in the interaction between heat waves and urban heat island intensity. Across 3,648 urban grid cells, heatwave conditions amplify the UHI markedly more at night. The \ensuremath{\Delta }UHI averages +0.27 \ensuremath{\pm} 0.23 \ensuremath{^\circ}C nocturnally. In contrast, the daytime interaction is negligible at +0.02 \ensuremath{\pm} 0.22 \ensuremath{^\circ}C. Spatially, this daytime effect exhibits significant heterogeneity, with nighttime \ensuremath{\Delta }UHI values exceeding 0.8 \ensuremath{^\circ}C in hotspots over the Indo-Gangetic and North China Plains. At the same time, humid coastal and equatorial regions show near-neutral or slightly negative interactions. Attribution analysis using a CatBoost machine learning model with SHAP explains the contributions of these drivers, revealing that net longwave radiation is the single most significant contributor, accounting for 36.5\% of the nighttime signal and 28.2\% by day. At night, sensible heat flux is the second-ranked driver (21.0\%), whereas during the day, specific humidity (19.5\%) and 10-m wind speed (17.2\%) are among the most significant contributors. Importantly, the influence attributed to these key drivers is directionally consistent with established research, validating our approach.

The dominant drivers vary by climate, with moisture dynamics causing high variability in tropical zones, while radiative and aerodynamic processes are more critical in arid and continental regions. These findings underscore the importance of tailoring effective adaptation strategies to local climate contexts and diurnal cycles, such as prioritizing green infrastructure in the tropics and cool roofs in arid regions. Furthermore, this study demonstrates a novel machine learning-based approach to attributing the spatial variability in UHI-heatwave interactions to specific biophysical drivers, offering a tool whose success opens avenues for future climate research. 