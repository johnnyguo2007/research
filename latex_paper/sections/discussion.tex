\section*{Discussion}




\subsection*{Detailed Analysis of Key Contributing Factors}



\subsubsection*{Humidity, Soil Moisture}Atmospheric and soil humidity significantly modulate the UHI-HW interaction by influencing the partitioning between sensible and latent heat fluxes. Figure~\ref{fig:global_hourly_key_feature_shap}b (specific humidity) and Figure~\ref{fig:global_hourly_key_feature_shap}c (soil moisture) show atmospheric and soil moisture decrease during HW periods compared to the background periods. Drier air in cities is a well-known amplifier of the UHI and HW synergy\unskip~\cite{2755510:33598950,2755510:33598924,2755510:33598925,2755510:33598926,2755510:33598952} . Observational and modeling studies report a consistent negative correlation between UHI intensity and atmospheric humidity\unskip~\cite{2755510:33598949,2755510:33598952,2755510:33598937} . Similarly, reduced soil moisture increases UHI-HW interaction by limiting evaporative cooling\unskip~\cite{2755510:33598950,2755510:33598925,2755510:33598926,2755510:33598915}. Rural landscapes, with more vegetation and permeable surfaces, retain more soil water than impervious urban areas\unskip~\cite{2755510:33598915,2755510:33598938}. During HWs, this contrast becomes emphasized: moist rural soils release latent heat through evapotranspiration, cooling the surface, whereas water-limited urban surfaces shift more energy into sensible and stored heat, raising urban temperatures and reinforcing the UHI HW synergy\unskip~\cite{2755510:33598924,2755510:33598925,2755510:33598926,2755510:33598935}. In our study, the decreases in specific humidity and soil moisture observed in Figure~\ref{fig:global_hourly_key_feature_shap}c and Figure~\ref{fig:global_hourly_key_feature_shap}e coincide with positive SHAP contributions to \ensuremath{\Delta }UHI, which is consistent with previous research.

Near sunrise, rural cooling can occur due to overnight dew that evaporates in the morning, thereby increasing the UHI intensity before rural areas warm rapidly\unskip~\cite{2755510:33598926}.  The big drop of \ensuremath{\Delta }\ensuremath{\delta }q\ensuremath{_{a}} and increase in positive SHAP contribution in Figure~\ref{fig:global_hourly_key_feature_shap}c and Figure~\ref{fig:global_hourly_key_feature_shap} e capture these diurnal shifts in humidity and soil moisture influence.

Our study, through the simulation and analysis of SHAP values, generally aligns with the existing literature in highlighting atmospheric humidity and soil moisture as key drivers of UHI HW synergies. 





\subsubsection*{Sensible and Latent Heat Flux}Sensible heat flux (SH) is the turbulent transfer of heat from the surface to the adjacent air, contributing to the warming of the atmosphere. Figure~\ref{fig:global_hourly_key_feature_shap} b illustrates a diurnal pattern of differences in sensible heat flux (\ensuremath{\Delta }SH) and its contribution to UHI HW interactions. \mbox{}\protect\newline During the daytime, \ensuremath{\Delta }SH are positive and thus positively contribute to the Urban Heat Island (UHI) intensity during HWs, as indicated by a positive SHAP contribution. This alignment with existing studies\unskip~\cite{2755510:33598941,2755510:33598935}  confirms reports that, during the daytime, HWs increase sensible heat flux in urban areas compared to rural areas and directly intensify the daytime urban heat island (UHI) effect\unskip~\cite{2755510:33598935} . Khan et al. (2020)\unskip~\cite{2755510:33598943}  further explain that higher surface temperatures during HWs increase ambient temperatures through convection, which subsequently elevates sensible heat flux. \mbox{}\protect\newline At nighttime, Figure~\ref{fig:global_hourly_key_feature_shap} b reveals negative differences in sensible heat flux between heatwave and non-heatwave conditions (\ensuremath{\Delta }SH). These negative differences result in negative SHAP contributions to UHI intensity. Li et al. (2015)\unskip~\cite{2755510:33598935}  corroborate this finding by similarly identifying negative nighttime sensible heat flux differences under heatwave conditions. Although nocturnal UHI intensity generally intensifies during heatwaves, studies frequently attribute this intensification primarily to the heat released from urban structures stored during the daytime and increased anthropogenic heat emissions, rather than changes in sensible heat flux\unskip~\cite{2755510:33598928} . The negative nighttime sensible heat flux differences can occur because rural temperatures cool down relatively more slowly than NHW periods at night, driven by diminished evapotranspiration and reduced soil moisture during HWs\unskip~\cite{2755510:33598893} . Yang and Zhao (2024)\unskip~\cite{2755510:33598892}  reinforce these observations and highlight that nocturnal UHI intensity responses to heat extremes predominantly depend on changes in surface heat storage, including urban-rural variations in surface albedo and impervious surface proportions, rather than sensible heat flux. In our study, the SHAP method effectively distinguishes these influences and demonstrates that overall UHI intensity may indeed increase at night. At the same time, individual components, such as sensible heat flux, exhibit negative differences and  \ensuremath{\Delta }UHI contributions.



\subsubsection*{Wind}Wind speed is one of the primary drivers of urban heat island intensity. Generally, low wind speeds intensify the UHII by reducing advective cooling from rural areas\unskip~\cite{2755510:33598893,2755510:33598924,2755510:33598926,2755510:33598952,2755510:33598927,2755510:33598930} . During HWs, these dynamics are amplified. HWs are often associated with stagnant, high-pressure systems that reduce regional wind speeds\unskip~\cite{2755510:33598926} . This reduction hinders the dispersion of urban heat, particularly at night, and increases atmospheric stability, creating synergistic HW UHI interactions\unskip~\cite{2755510:33598950,2755510:33598926,2755510:33598891,2755510:33598941,2755510:33598930} . 

However, these interactions exhibit regional variability. For instance, a study in Singapore found no significant reduction in wind speed or UHI-HW synergy during a heatwave\unskip~\cite{2755510:33598915} . In Portland, nighttime wind speeds actually increased during HWs, which was associated with a lower UHI intensity\unskip~\cite{2755510:33598921} . In Seoul, nighttime wind speed was a complex yet significant predictor of UHI intensity during HWs, even when non-HW correlations were weak\unskip~\cite{2755510:33598952} .

Our model's analysis of the 10 m wind (U10)  feature aligns with established research, particularly in the context of nocturnal \ensuremath{\Delta }UHI contributions. During the nighttime, the SHAP beeswarm plot (Figure~\ref{fig:shap_summary_day_night}) shows a strong inverse relationship where low wind speeds increase predicted \ensuremath{\Delta }UHI while high speeds mitigate it. From Figure~\ref{fig:global_hourly_key_feature_shap} d, we can also observe the negative co-movement between the U10 value and its SHAP contribution.  During the daytime,  both Figure~\ref{fig:shap_summary_day_night}a and Figure~\ref{fig:global_hourly_key_feature_shap}d exhibit a weak negative co-movement. It is fair to say our model does not show a consistent directional contribution during the daytime.





\subsection*{K{\"{o}}ppen{\textendash}Geiger climate zones discussion}UHI-HW interactions vary significantly across K{\"{o}}ppen-Geiger climate zones (Figures~\ref{fig:diurnal_duhi_by_climate_zone} and~\ref{fig:duhi_variance_by_climate_zone}), highlighting the influence of the background climate. Our analysis identifies a key distinction between two dominant mechanisms that govern these interactions: moisture dynamics in Tropical zones and radiative and aerodynamic processes in Arid and Continental zones. \mbox{}\protect\newline In Tropical climates, moisture serves as the primary driving force. These regions show the most variability in the UHI between heatwave (HW) and non-heatwave (NHW) conditions(Figure~\ref{fig:duhi_variance_by_climate_zone}). This variability primarily arises from changes in rural moisture availability during HWs. Initially, tropical regions have abundant humidity and soil moisture, which enables substantial evaporative cooling in rural areas and thus keeps the UHI relatively subdued. However, as a heatwave continues, rural moisture stores become depleted(Figure~\ref{fig:daily_evolution_duhi_by_climate}). This depletion weakens evaporative cooling, a process not significant in urban areas due to their impervious surfaces. The gradual loss of cooling amplifies the UHI, increasing the variance of \ensuremath{\Delta }UHI. At the intraday level, the SHAP analysis highlights specific humidity (q\ensuremath{_{a}}) and turbulent fluxes as the primary contributors in Tropical regions(Figure~\ref{fig:feature_contribution_by_climate_zone}). The sharp positive SHAP value from specific humidity in the morning (Figure~\ref{fig:q2m_shap_by_climate_zone}d) notably captures the moment when rural moisture decreases enough to intensify the UHI. \mbox{}\protect\newline Arid and Continental climates rely predominantly on radiative and aerodynamic mechanisms. In these zones, the UHI-HW interactions are more consistent (Figure~\ref{fig:duhi_variance_by_climate_zone}). This stability occurs because these areas are already moisture-limited at the beginning of the heat waves, resulting in little change in rural evaporative cooling (Figure~\ref{fig:daily_evolution_duhi_by_climate}). Without the material contribution from latent heat flux, other processes take precedence. Specifically, wind speed (U10) gains importance (Figure~\ref{fig:feature_contribution_by_climate_zone}). In conditions where evaporative cooling cannot buffer temperatures, reduced wind speeds during stagnant heat waves amplify the UHI effect in these climate zones even more visibly.

In summary, the background climate influences the dominant physical mechanisms underlying the interaction between UHII and HWs. In moisture-rich tropical regions, this interaction is significant and dynamic, characterized by the temporary depletion of moisture in rural areas, resulting in high variability. In contrast, in water-limited arid and continental climates, the interaction is mainly determined by the static properties inherent to urban environments, such as radiative trapping and sensitivity to changes in wind advection. This fundamental distinction underscores the necessity for climate-specific strategies to mitigate UHI, as measures effective in one climatic context may not be equally applicable or successful in another.



\subsection*{ \mbox{}\protect\newline  Limitations and need for future research}CLM5 does not account for lateral heat and moisture transport between urban and rural land units. This limitation may result in an overestimation of urban-rural microclimate gradients, particularly for smaller cities.\unskip~\cite{2755510:33598947}  While the SHAP\unskip~\cite{2755510:33598933} method provides valuable insights into feature importance, its stability can be influenced by factors such as multicollinearity and the choice of background dataset; therefore, results should be interpreted with this in mind. Further research should investigate the use of coupled urban-atmosphere models to represent these lateral transport processes. Investigating the impact of different urban morphologies and land cover types within each climate zone would also provide valuable insights. Future work could also focus on developing downscaled climate projections to assess the evolution of UHI-HW interactions under different climate change scenarios. Finally, incorporating socioeconomic factors and human behavioral responses into the modeling framework would improve the realism and applicability of the results for urban heat adaptation planning. 